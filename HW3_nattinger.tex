% !TEX TS-program = pdflatex
% !TEX encoding = UTF-8 Unicode

% This is a simple template for a LaTeX document using the "article" class.
% See "book", "report", "letter" for other types of document.

\documentclass[11pt]{article} % use larger type; default would be 10pt

\usepackage[utf8]{inputenc} % set input encoding (not needed with XeLaTeX)

%%% PAGE DIMENSIONS
\usepackage{geometry} % to change the page dimensions
\geometry{a4paper} % or letterpaper (US) or a5paper or....

\usepackage{graphicx} % support the \includegraphics command and options

\usepackage{amssymb}
\usepackage{amsmath}
%%% PACKAGES
\usepackage{booktabs} % for much better looking tables
\usepackage{array} % for better arrays (eg matrices) in maths
\usepackage{paralist} % very flexible & customisable lists (eg. enumerate/itemize, etc.)
\usepackage{verbatim} % adds environment for commenting out blocks of text & for better verbatim
\usepackage{subfig} % make it possible to include more than one captioned figure/table in a single float
% These packages are all incorporated in the memoir class to one degree or another...

%%% HEADERS & FOOTERS
\usepackage{fancyhdr} % This should be set AFTER setting up the page geometry
\pagestyle{fancy} % options: empty , plain , fancy
\renewcommand{\headrulewidth}{0pt} % customise the layout...
\lhead{}\chead{}\rhead{}
\lfoot{}\cfoot{\thepage}\rfoot{}

%%% SECTION TITLE APPEARANCE
\usepackage{sectsty}
\allsectionsfont{\sffamily\mdseries\upshape} % (See the fntguide.pdf for font help)
% (This matches ConTeXt defaults)

%%% ToC (table of contents) APPEARANCE
\usepackage[nottoc,notlof,notlot]{tocbibind} % Put the bibliography in the ToC
\usepackage[titles,subfigure]{tocloft} % Alter the style of the Table of Contents
\renewcommand{\cftsecfont}{\rmfamily\mdseries\upshape}
\renewcommand{\cftsecpagefont}{\rmfamily\mdseries\upshape} % No bold!

\usepackage{amsmath}
\DeclareMathOperator*{\argmax}{arg\,max}
\DeclareMathOperator*{\argmin}{arg\,min}

%%% END Article customizations

%%% The "real" document content comes below...

\title{HW3}
\author{Michael B. Nattinger\footnote{I worked on this assignment with my study group: Alex von Hafften, Andrew Smith, Ryan Mather, and Tyler Welch. I have also discussed problem(s) with Emily Case, Sarah Bass, and Danny Edgel.}}

%\date{} % Activate to display a given date or no date (if empty),
         % otherwise the current date is printed 

\begin{document}
\maketitle

\section{Question 1}
Let $(X, d)$ be a nonempty complete metric space. Define $T: X \rightarrow X$ such that $d(T(x),T(y))<d(x,y)$ $\forall x \neq y, x,y \in X.$ I will prove by counterexample that it is not necessarily the case that $T$ has a fixed point.

\underline{pf} Consider $(\mathbb{R}^{+},d)$ with $d(x,y) = |x-y|$. Define $T: \mathbb{R}^{+} \rightarrow \mathbb{R}^{+}$ as $T(x) = \sqrt{x^2 + 1}$. Let $x,y \in \mathbb{R}^{+}$ such that $x \neq y$. Then $d(T(x),T(y)) = |\sqrt{x^2 + 1} - \sqrt{y^2 +1}| < |\sqrt{x^2} - \sqrt{y^2}| = |x - y| = d(x,y)$.\footnote{ $ |\sqrt{x^2 + 1} - \sqrt{y^2 +1}| = \frac{|x^2 + y^2|}{|\sqrt{x^2 + 1} + \sqrt{y^2 +1}|}  = |\sqrt{x^2} - \sqrt{y^2}| \frac{|\sqrt{x^2} + \sqrt{y^2}|}{|\sqrt{x^2 + 1} + \sqrt{y^2 + 1}|} \leq |\sqrt{x^2} - \sqrt{y^2}|$, with equality iff $x=y=0.$ However, $x \neq y,$ so $|\sqrt{x^2 + 1} - \sqrt{y^2 +1}| < |\sqrt{x^2} - \sqrt{y^2}|$. } Furthermore, $\mathbb{R}^{+} \subset \mathbb{R}$ is complete as it is closed and $\mathbb{R}$ is complete. We will show that $T$ has no fixed points.

Assume for the purpose of contradiction that $T$ has a fixed point, $x^*$. Then $x^* = T(x^*) = \sqrt{(x^*)^2 + 1} > \sqrt{(x^*)^2} = x^*$, so $x^* > x^*$, which is a contradiction. So, $T$ has no fixed points.

% Let $x_0 \in X$ be arbitrary. We will then define $\{ x_n\}$ s.t. $x_n = T(x_{n-1})$ $\forall x \in \mathbb{N}.$ Then $d(x_{n+1},x_{n}) = d(T(x_{n}),T(x_{n-1})) < d(x_{n},x_{n-1})$ $\forall n \in \mathbb{N}$. So, by the monotone convergence theorem, $\{ a_n \}$ defined as $a_n = d(x_n,x_{n-1})$ converges to its infimum. Its infimum is 0.\footnote{How do I prove this?} Thus, for $\epsilon > 0$ we can find $N \in \mathbb{N}$ such that $\forall n>N,$ $|d(x_n,x_{n-1}) - 0| = d(x_n,x_{n-1})<\epsilon$ so $\{ x_n\}$ is Cauchy. Since X is complete, we conclude that $\{ x_n\}$ converges to some limit and define $\lim\limits_{n \rightarrow \infty}x_n = x$. Then, note $T(x) = T(\lim\limits_{n \rightarrow \infty} x_n) = \lim\limits_{n \rightarrow \infty} T(x_n) = \lim\limits_{n \rightarrow \infty} x_{n+1} = x$.\footnote{This holds because T is continuous which I will prove here: \\ Let $x \in X$ be arbitrary. Let $\epsilon >0.$ Then for $\delta = \epsilon$, $|T(x) - T(y)| < |x - y|$ $\forall y \in \mathbb{R}$ such that $|x-y|<\delta$. Thus, $T$ is continuous.} Thus, $x$ is a fixed point of $T$.
%
%Assume for the purpose of contradiction that $\exists y \in X \setminus \{x \}$ s.t. $y$ is a fixed point. Then $d(T(x),T(y)) < d(x,y)$ since $x\neq y$. However, $T(x) = x$ and, by assumption, $T(y) = y$, so $d(T(x),T(y)) = d(x,y)$, a contradiction. Thus, $x$ is the unique fixed point of $T$.

%Outline:  Manipulate $d(T(x),T(y))<d(x,y)$ and show it is continuous on X. Then X is closed so we can apply EVT and get a $\beta$ that we can use in the central mapping theorem.
%
%Define $f(x,y) = d(x,y) - d(T(x),d(T(y)).$ 

% Since $d(T(x),T(y))<d(x,y)$ $\forall x \neq y, x,y \in X,$ $\exists \beta <1$ such that $d(T(x),T(y))\leq \beta d(x,y)<d(x,y)$ $\forall x \neq y, x,y \in X.$ Then T has a unique fixed point $x^*$ by the contraction mapping theorem.

\section{Question 2}
We will show that the following countable set, $A$, is compact in $\mathbb{R}$: $A:= \{ \frac{1}{n}: n \in \mathbb{N}\} \cup \{ 0 \}$.

\underline{pf} We can map each element of $A$ to an element of $\mathbb{N}$. We can do so by mapping 0 to 1, and by mapping $\frac{1}{n}$ to $n+1$ for $n \in \mathbb{N}, n>1.$ Thus $A$ is countable.

Given any open cover of $A$, we know that $\exists$ an open set $S$ of the cover which contains 0. As $S$ is open, and since the sequence $\{\frac{1}{n}\}_{n \in \mathbb{N}} \rightarrow 0,$ $S$ contains an infinite number of elements of $A$. Formally, $\exists N \in \mathbb{N}$ s.t. $\forall n>N, \{\frac{1}{n}\}\subset S$. We then know that there can be at most $N$ elements of $A$ not contained in $S$, so we can form a finite open cover of $A$ through a union of $S$ with at most $N$ more open sets of the cover, one containing a cover of $\frac{1}{1}$, one containing a cover of $\frac{1}{2}$, $\dots$, one containing a cover of $\frac{1}{N}$. Thus, we can form a finite subcover of $A$, so $A$ is compact.
%We can map each element of $A$ to an element of $\mathbb{N}$. We can do so by mapping 0 to 1, and by mapping $\frac{1}{n}$ to $n+1$ for $n \in \mathbb{N}, n>1.$ Thus $A$ is countable. Additionally $A$ is bounded below by its smallest element, 0, and above by its largest element, 1. So $A$ is bounded.


%We will now prove that $A$ is closed. Let $\{ a_n \}$ be a convergent subsequence, with $a_n \in A$ $\forall n \in \mathbb{N}$. Let $\lim\limits_{n\rightarrow \infty} a_n = a$. Assume for purpose of contradiction that $a \notin A$. We then have three possible cases:
%\begin{enumerate}
%\item Case 1: $a<0$. By the definition of convergence, for $\epsilon = \frac{a}{2}$, $\exists N \in \mathbb{N}$ s.t. $\forall n>N,$ $|a_n - a|<\epsilon.$ However, all elements of $A$ are nonnegative, and the convergence of $\{a_n\}$ implies the existence of an element of $A$, $a_{n^*}$, s.t. $a_{n^*}< \frac{a}{2}<0$, a contradiction.
%\item Case 2: $a>1$.  By the definition of convergence, for $\epsilon = \frac{a-1}{2}$, $\exists N \in \mathbb{N}$ s.t. $\forall n>N$, $|a_n - a|<\epsilon.$ Note however that $\max A =1$, and the convergence of $\{a_n\}$ implies the existence of an element of $A$, $a_{n^*}$, s.t. $1<a-\epsilon<a_{n^*}$, a contradiction.
%\item Case 3: $a \in [0,1]\setminus A$. Then we can find $n^*:= \argmin\limits_{n \in \mathbb{N}} |a - \frac{1}{n}|$. We can then define $\epsilon = \frac{|a - \frac{1}{n^*}|}{2}$, and by the definition of convergence $\exists N \in \mathbb{N}$ s.t. $\forall n>N,$ $|a_n - a|<\epsilon$.\footnote{Since $a \in [0,1] \setminus A$, and 0 is in A, the closest element of A to a must be nonzero as $a>0$ so we can always find some sufficiently large $n\in \mathbb{N}$ with $0<\frac{1}{n}\leq a$. Since we have defined this epsilon ball such that it is closer to $a$ than the closest element of $A$ is to $a$, 0 cannot be the $a_n \in A$ that satisfies this inequality.} So, for some $\tilde{n} \in \mathbb{N}, |\frac{1}{\tilde{n}} -a|<\epsilon$, yet by construction $|a_{n^*} - a|>\frac{|a - \frac{1}{n^*}|}{2}=\epsilon$. This implies that $|a - \frac{1}{n^*}| = \min\limits_{n \in \mathbb{N}} |a - \frac{1}{n}| > |a - \frac{1}{\tilde{n}}|$, which is a contradiction. %Then $\frac{1}{n} \in A$
%\end{enumerate}
%Thus, as all possible cases lead to contradictions, $a \in A$ so $A$ is closed. Since $A$ is closed and bounded it is compact, and as we showed earlier $A$ is countable so $A$ is an example of a countable set that is compact in $\mathbb{R}$.
%Notice that the smallest element of $S$ is $0+0=0$ and the largest element is $1+1 = 2$. Thus, S is bounded. Now we just need that it is closed.
%
%Show that each element of $H$ is a limit point of S. (ERGO IT CONTAINS ALL ITS LIMIT POINTS)
%
%S is closed because it is the sum of two countable subsets of $\mathbb{R}$ and is therefore closed in $\mathbb{R}$, and is therefore compact because it is a closed and bounded subset of $\mathbb{R}$.

\section{Question 3}
Prove that the function $f(x) = \cos^2 (x)e^{5-x-x^2}$ has a maximum on $\mathbb{R}$.

\underline{pf} Let $g(x):= \cos^2(x)$, $h(x):= e^{5-x-x^2}$. Thus $f(x) = g(x)h(x)$ $\forall x \in \mathbb{R}.$ $\cos$ and exp are continuous functions so $f(x),g(x),h(x)$ are all continuous on $\mathbb{R}$. Note that $0\leq g(x) \leq 1$ $\forall n \in \mathbb{N}.$ Note also that $0<h(x) \Rightarrow 0 \leq f(x) $ $\forall x \in \mathbb{R}$. Furthermore, for $x \in (-\infty,-3) \cup (3,\infty)$, $h(x) < 1 \Rightarrow f(x) \leq 1$.\footnote{For $x \in (-\infty,-3) \cup (3,\infty),$ $5-x-x^2<0 \Rightarrow e^{5 - x - x^2} = h(x) < 1.$ Assume for the purpose of contradiction that $f(x^*)>1$ for some $x^* \in (-\infty,-3) \cup (3,\infty).$ This implies $g(x^*)h(x^*) > 1$, so either $g(x^*)>1$ or $h(x^*) >1.$ In either case we have a contradiction so $f(x)\leq1$ $\forall x \in (-\infty,-3) \cup (3,\infty).$} Notice that $f(0) = \cos^2(0)e^{5 - 0 - 0^2} = (1)(e^5) > 1$. Notice also that $[-3,3]$ is compact in $\mathbb{R}$ by the Heine-Borel theorem, so by the extreme value theorem $\exists k \in [-3,3]$ s.t. $f(k) \geq f(x)$ $\forall x \in [-3,3]$. Therefore, $f(k) \geq f(0) > 1 \geq f(x)$ $\forall x \in (-\infty,-3) \cup (3,\infty)$, so $f(k) \geq f(x)$ $\forall x \in \mathbb{R}.$ Thus, $f(x)$ has a maximum on $\mathbb{R}$, at $k$.

% Maybe an extention of the extended case of the extreme value theorem?
%$f(x):=\cos^2(x)$ is continuous and bounded by 0 and 1, and $g(x):=e^{5 - x - x^2}$ is continuous so it is sufficient to show that it  bounded by 0 and some upper bound M, as then their product must be bound by 0 and $M$.\footnote{Let $f(x),g(x): I \rightarrow \mathbb{R}$ be continuous functions, with $f(x)$ bounded by 0 and 1 and $g(x)$ bounded by 0 and $M$, and with $I \subseteq \mathbb{R}$ a closed interval. Then let $h(x):=$ $f(x)g(x)$. Assume $h(x)$ is not bounded below by 0. Then $\exists x \in I$ s.t. $f(x)g(x)<0$ which implies that either $f(x)<0$ or $g(x)<0$, either of which is a contradiction. Assume that $h(x)$ is not bounded above by $M$. This implies that $\exists x \in I$ s.t. $f(x)g(x) > M$, so either $f(x)>1$ or $g(x)>M$, either of which is a contradiction. So, $h(x)$ is bounded by 0 and $M$.} Note that $f(x)$ is increasing on $(-\infty,-1)$ and decreasing on $(0,\infty)$,\footnote{prove inc, dec here}. Note that $[-1,0]$ is compact on $\mathbb{R}$ by the Heine-Borel theorem. Then, by the extreme value theorem, $g(x)$ achieves its maximum for some  on $[-1,0]$ and is thus bounded above by some $M \geq 0$. Therefore, $h(x)$ is bounded above by $M$ and below by $0$, and this   %Show e^(...) is decreasing below a certain point and increasing above a certain point so we can just use 

\section{Question 4}
Suppose you have two maps of Wisconsin, one large and one small. We put the large one on top of the small one so that the small map is completely covered by the large one. Prove that a point on the small map is in the same location as it is on the large map.

\underline{pf} Let the large (flat) map of Wisconsin = $W$ be a closed subset of $\mathbb{R}^2$. Since $\mathbb{R}^2$ is complete and $W$ is closed, $W$ is complete. We then have a strictly smaller map of Wisconsin, laid on top of the large map such that it is completely covered by the large map, and so we can define a mapping operator $T: W \rightarrow W$, described thusly: 

Take an arbitrary $x \in W$. This is a point somewhere on the big map, and hence corresponds to some physical location in Wisconsin. There is a point on the small map of Wisconsin which corresponds to the same geographical location. The small map is covered by the large map, so this point on the small map is touching a point on the large map. The point on the large map that touches the point on the small map corresponding to the same location as $x$ is $T(x)$.

Since by assumption the small map is smaller than the large map, $\exists \beta <1$ such that $\forall x,y \in W$ we have $\beta d(x,y)\geq d(T(x),T(y)).$\footnote{This is because the d(T(x),T(y)) will be the same as the distance on the small map between the points corresponding to the same locations as x and y correspond to on the large map. The small map is shrunk by some $\beta <1$ relative to the big map, so $\beta d(x,y) = d(T(x),T(y))$ } Then $T$ is a contraction of modulus $\beta$, so by the contraction mapping theorem $T$ has a fixed point $x^*$ where $x^* = T(x^*)$ so at $x^*$ the big map and small map correspond to the same location in Wisconsin.

\section{Question 5}
Consider the set $X = \{ -1, 0, 1 \}$ and the space of all functions on $X$, $F_X = \{ f: X \rightarrow \mathbb{R} \}$.

\subsection{Show that $F_X$ is a vector space.}
\subsection{Show that the operator $T: F_X \rightarrow F_X$ defined by $T(f)(x) = f(x^2), x \in \{ -1,0,1\}$ is linear.}
\subsection{Calculate kern $T$, Im $T$, and rank $T$.}
\section{Question 6}
Consider the following system of linear equations:
\begin{equation}
\begin{cases}
0=& x_1 + x_2 + 2x_3 + x_4, \\
0=&3x_1 - x_2 + x_3 - x_4, \\
0=&5x_1 - 3x_2 - 3x_4. \label{eqn:linsys}
\end{cases}
\end{equation}
Let $X$ be the set of $\{ x_1,x_2,x_3,x_4 \}$ which satisfy (\ref{eqn:linsys}).
\subsection{Show that $X$ is a vector space.}
\subsection{Calculate dim $X$.}
\end{document}
