% !TEX TS-program = pdflatex
% !TEX encoding = UTF-8 Unicode

% This is a simple template for a LaTeX document using the "article" class.
% See "book", "report", "letter" for other types of document.

\documentclass[11pt]{article} % use larger type; default would be 10pt

\usepackage[utf8]{inputenc} % set input encoding (not needed with XeLaTeX)

%%% PAGE DIMENSIONS
\usepackage{geometry} % to change the page dimensions
\geometry{a4paper} % or letterpaper (US) or a5paper or....

\usepackage{graphicx} % support the \includegraphics command and options

\usepackage{amssymb}
\usepackage{amsmath}
%%% PACKAGES
\usepackage{booktabs} % for much better looking tables
\usepackage{array} % for better arrays (eg matrices) in maths
\usepackage{paralist} % very flexible & customisable lists (eg. enumerate/itemize, etc.)
\usepackage{verbatim} % adds environment for commenting out blocks of text & for better verbatim
\usepackage{subfig} % make it possible to include more than one captioned figure/table in a single float
% These packages are all incorporated in the memoir class to one degree or another...

%%% HEADERS & FOOTERS
\usepackage{fancyhdr} % This should be set AFTER setting up the page geometry
\pagestyle{fancy} % options: empty , plain , fancy
\renewcommand{\headrulewidth}{0pt} % customise the layout...
\lhead{}\chead{}\rhead{}
\lfoot{}\cfoot{\thepage}\rfoot{}

%%% SECTION TITLE APPEARANCE
\usepackage{sectsty}
\allsectionsfont{\sffamily\mdseries\upshape} % (See the fntguide.pdf for font help)
% (This matches ConTeXt defaults)

%%% ToC (table of contents) APPEARANCE
\usepackage[nottoc,notlof,notlot]{tocbibind} % Put the bibliography in the ToC
\usepackage[titles,subfigure]{tocloft} % Alter the style of the Table of Contents
\renewcommand{\cftsecfont}{\rmfamily\mdseries\upshape}
\renewcommand{\cftsecpagefont}{\rmfamily\mdseries\upshape} % No bold!

\usepackage{amsmath}
\DeclareMathOperator*{\argmax}{arg\,max}
\DeclareMathOperator*{\argmin}{arg\,min}

\newcount\colveccount
\newcommand*\colvec[1]{
        \global\colveccount#1
        \begin{pmatrix}
        \colvecnext
}
\def\colvecnext#1{
        #1
        \global\advance\colveccount-1
        \ifnum\colveccount>0
                \\
                \expandafter\colvecnext
        \else
                \end{pmatrix}
        \fi
}

%%% END Article customizations

%%% The "real" document content comes below...

\title{HW3}
%\author{Michael B. Nattinger\footnote{I worked on this assignment with my study group: Alex von Hafften, Andrew Smith, Ryan Mather, and Tyler Welch. I have also discussed problem(s) with Emily Case, Sarah Bass, and Danny Edgel.}}
\author{Michael B. Nattinger}

%\date{} % Activate to display a given date or no date (if empty),
         % otherwise the current date is printed 

\begin{document}
\maketitle

\section{Real Analysis}
De Morgan's Laws: $(A \cap B)^c = A^c \cup B^c$; $(A \cup B)^c = A^c \cap B^c$.

The \textbf{cardinality} of a set is the size of the set. Two sets are \textbf{numerically equivalent} if they have the same cardinality. A set is \textbf{countably infinite} if it is numerically equivalent to $\mathbb{N}$.

A \textbf{metric} on a set $X$ is a function $d: X $ x $ X \rightarrow \mathbb{R}^+$ s.t. $\forall x,y,z \in X,$
\begin{itemize}
\item $d(x,y) \geq 0$, with equality $\iff x = y;$
\item $d(x,y) = d(y,x)$;
\item $d(x,z) \leq d(x,y) + d(y,z).$
\end{itemize}
A \textbf{metric space} is a pair $(X,d),$ where $X$ is a set and $d$ is a metric on $X$. Examples include Euclidean space.

In a metric space, $(X,d),$ an \textbf{open ball} is $B_\epsilon(x) = \{ y \in X| d(x,y) < \epsilon\}$ and a \textbf{closed ball} is $B_\epsilon[x] = \{ y \in X| d(x,y) \leq \epsilon\}$.

A \textbf{sequence} in a set $X$ is a function $s: \mathbb{N} \rightarrow X,$ denoted $\{ s_n\}$, where $s_n = s(n)$. A sequence $x_n$ in a metric space $(X,d)$ \textbf{converges} to $x \in X$ if $\forall \epsilon >0, \exists N(\epsilon)>0$ s.t. $\forall n>N(\epsilon)$ $d(x_n,x) < \epsilon.$ We write $x_n \rightarrow x$ or $\lim\limits_{n \rightarrow \infty} x_n = x.$

A sequence in a metric space has \textbf{at most one limit}.

Consider a sequence $\{ x_n\}$ and a rule that assigns to each $k \in \mathbb{N}$ a value $n_k \in \mathbb{N}$ such that $n_k <n_{k+1}$ for all $k$. Then $\{ x_{n_k}\}$ is a \textbf{subsequence}. If $\{x_n\} \rightarrow x$ then any subsequence also converges to $x$.

A subset $s \subset X$ in a metric space $(X,d)$ is \textbf{bounded} if $\exists x \in X, \beta \in \mathbb{R} s.t. \forall s \in S, d(x,s) < \beta.$ Every convergent sequence in a metric space is bounded.

Limits preserve weak inequality.

If $x_n \rightarrow x, y_n \rightarrow y$, $x_n + y_n \rightarrow x + y$, $x_ny_n \rightarrow xy, x_n/y_n \rightarrow x/y$ so long as $y_n,y$ nonzero. Same applies to $\mathbb{R}^n$ with operations taken coordinate-wise.

\textbf{Bolzano-Weierstrass Theorem}: Every bounded real sequence contains at least one convergent subsequence.

\textbf{Monotone Convergence Theorem}: Every increasing sequence of real numbers that is bounded above converges. Every decreasing sequence of real numbers that is bounded below converges.

Every real sequence contains either a decreasing subsequence or increasing subsequence or possible both.

Given a real sequence $\{ x_n\}$, the infinite sum of its terms is well-defined if the sequence of partial sums converges.

Let $(X,d)$ be a metric space. A set $A\subset X$ is \textbf{open} if $\forall x \in A \exists \epsilon > 0 s.t. B_\epsilon(x) \subset A.$ A set $C \subset X$ is \textbf{closed} if its complement is open. This depends on the metric space. $[0,1]$ is not open in $(\mathbb{R},d_E)$ but is open in $([0,1],d_E)$.

Let $X,d)$ be a metric space.
\begin{itemize}
\item $\emptyset,X$ are simultaneously open and closed in X;
\item the union of an arbtrary collection of open sets is open;
\item the intersection of a finite collection of closed sets is closed;
\item the union of a finite collection of closed sets is closed;
\item the intersection of an arbitrary collection of closed sets is closed.
\end{itemize}
$\bigcup_{n=1}^{\infty}\left[ \frac{1}{n}, 1 - \frac{1}{n}\right] = (0,1)$, $\bigcap_{n=1}^{\infty}\left(1- \frac{1}{n}, 1 + \frac{1}{n}\right) = \{1\}$

A set is closed if and only if every convergent sequence contained in A has its limit in A.

Let $(X,d)$, and $A \in X$. $x \in X$ is a \textbf{limit point} of A if $\forall \epsilon > 0,$ $(B_{\epsilon}(x) \setminus \{ x\})\cap A \neq \emptyset.$

Let $(X,d),(Y,\rho)$ be two metric spaces, $A \subset X, f:A \rightarrow Y, x_0 =$limit point of $A$. $f$ has a limit $y_0$ as $x$ approaches $x_0$ if $\forall \epsilon>0 \exists \delta > 0 s.t.$ if $x \in A$ and $0<d(x,x_0) < \delta,$ then $\rho(f(x),y_0) < \epsilon.$

$\lim\limits_{x \rightarrow x_0} f(x) = y_0$ if and only if for any sequence $\{ x_n\} \in X$ such that $x_n \rightarrow x_0$ and $x_n\neq x_0$, the sequence $\{ f(x_n)\}$ converges to $y_0$.

The limit of $f$ as $x \rightarrow x_0$, when it exists, is unique.

A function is \textbf{continuous} at $x_0$ if $\forall \epsilon>0 \exists \delta>0$ s.t. if $d(x,x_0) < \delta$, then $\rho(f(x),f(x^0)) < \epsilon$. ($\delta$ can vary for different $x^0$ and $\epsilon$)

A function $f$ is continuous at $x_0$ if and only if one of the following equivalent statements is true:
\begin{itemize}
\item $f(x_0)$ is defined and either $x_0$ is an isolated point or $x_0$ is a limit point of X and $\lim\limits_{x \rightarrow x_0} f(x) = f(x_0)$.
\item For any sequence $\{ x_n\}$ s.t. $x_n \rightarrow x_0$, the sequence $\{ f(x_n)\}$ converges to $f(x_0)$.
\end{itemize}

A function $f$ is continuous if it is continuous at every point of its domain.

A function $f$ is continuous iff for any closed set C, the set $f^{-1}(C)$ is closed. A function $f$ is continuous iff for any open set A, the set $f^{-1}(A)$ is open.

A function is uniformly continuous if $\forall \epsilon > 0 \exists \delta>0$ s.t. $if d(x,x_0)<\delta,$ then $\rho(f(x),f(x_0)) < \epsilon$. Note: delta depends only on epsilon!

Uniform continuity implies continuity.

Let $(X,d),(Y,\rho)$ be two metric spaces, $f:X \rightarrow Y, E\subset X.$ Then f is \textbf{Lipschitz} on $E$ if $\exists K>0$ s.t. $\rho(f(x),f(y)) \leq K d(x,y) \forall x,y \in E.$ $f$ is \textbf{locally Lipschitz} on $E$ if $\forall x \in E \exists \epsilon > 0$ s.t. f is Lipscitz on $B_{\epsilon}(x) \cap E.$

Lipschitz implies uniform continuity.

Let $X \subset \mathbb{R}.$ Then $u \in \mathbb{R}$ is an upper bound for $X$ if $x\leq u \forall x \in X$ (and opposite for lower bound). X is bounded above if there is an upper boound for $X$.

Suppose $X$ is bounded above. The supremum of $X$, $sup X$, is the smallest upper bound for $X$, i.e. $supX$ satisfies
\begin{itemize}
\item $supX \geq x$ $\forall x \in X$;
\item $\forall y<supX \exists x \in X$ s.t. $x>y$.
\end{itemize}
 And infimum is similarly defined.

\textbf{Supremum Property}: Every nonempty set of real numbers that is bounded above has a supremum, and the supremum is a real number. (Not generally the case for all numbers e.g. sets that would be bounded by irrational numbers in the reals do not have a supremum when they are instead defined in the rationals)

\textbf{Extreme Value Theorem}: Let $f:[a,b]\rightarrow \mathbb{R}$ be continuous. Then $f$ attains its maximum and minimum on $[a,b]$.

\textbf{Intermediate Value Theorem}: Let $f:[a,b]\rightarrow \mathbb{R}$ be a continuous function. Then for any $\gamma \in [f(a),f(b)]$ there exists $c \in [a,b]$ s.t. $f(c) = \gamma$.

Let $f$ be monotonically increasing. Then one-sided limits exist for all x. Moreover, $sup\{f(s)|a<s<x\} = f(x^{-} \leq f(x) \leq f(x^{+}) = inf\{ f(s)|x<s<b\}$.

\section{Linear Algebra}

\end{document}
