% !TEX TS-program = pdflatex
% !TEX encoding = UTF-8 Unicode

% This is a simple template for a LaTeX document using the "article" class.
% See "book", "report", "letter" for other types of document.

\documentclass[11pt]{article} % use larger type; default would be 10pt

\usepackage[utf8]{inputenc} % set input encoding (not needed with XeLaTeX)

%%% PAGE DIMENSIONS
\usepackage{geometry} % to change the page dimensions
\geometry{a4paper} % or letterpaper (US) or a5paper or....

\usepackage{graphicx} % support the \includegraphics command and options

\usepackage{amssymb}
\usepackage{amsmath}
%%% PACKAGES
\usepackage{booktabs} % for much better looking tables
\usepackage{array} % for better arrays (eg matrices) in maths
\usepackage{paralist} % very flexible & customisable lists (eg. enumerate/itemize, etc.)
\usepackage{verbatim} % adds environment for commenting out blocks of text & for better verbatim
\usepackage{subfig} % make it possible to include more than one captioned figure/table in a single float
% These packages are all incorporated in the memoir class to one degree or another...

%%% HEADERS & FOOTERS
\usepackage{fancyhdr} % This should be set AFTER setting up the page geometry
\pagestyle{fancy} % options: empty , plain , fancy
\renewcommand{\headrulewidth}{0pt} % customise the layout...
\lhead{}\chead{}\rhead{}
\lfoot{}\cfoot{\thepage}\rfoot{}

%%% SECTION TITLE APPEARANCE
\usepackage{sectsty}
\allsectionsfont{\sffamily\mdseries\upshape} % (See the fntguide.pdf for font help)
% (This matches ConTeXt defaults)

%%% ToC (table of contents) APPEARANCE
\usepackage[nottoc,notlof,notlot]{tocbibind} % Put the bibliography in the ToC
\usepackage[titles,subfigure]{tocloft} % Alter the style of the Table of Contents
\renewcommand{\cftsecfont}{\rmfamily\mdseries\upshape}
\renewcommand{\cftsecpagefont}{\rmfamily\mdseries\upshape} % No bold!

\usepackage{amsmath}
\DeclareMathOperator*{\argmax}{arg\,max}
\DeclareMathOperator*{\argmin}{arg\,min}

\newcount\colveccount
\newcommand*\colvec[1]{
        \global\colveccount#1
        \begin{pmatrix}
        \colvecnext
}
\def\colvecnext#1{
        #1
        \global\advance\colveccount-1
        \ifnum\colveccount>0
                \\
                \expandafter\colvecnext
        \else
                \end{pmatrix}
        \fi
}

%%% END Article customizations

%%% The "real" document content comes below...

\title{Exam Note Sheet}
%\author{Michael B. Nattinger\footnote{I worked on this assignment with my study group: Alex von Hafften, Andrew Smith, Ryan Mather, and Tyler Welch. I have also discussed problem(s) with Emily Case, Sarah Bass, and Danny Edgel.}}
\author{Michael B. Nattinger}

%\date{} % Activate to display a given date or no date (if empty),
         % otherwise the current date is printed 

\begin{document}
\maketitle

\section{Real Analysis}
\subsection{Theorems}

\textbf{Bolzano-Weierstrass Theorem}: Every bounded real sequence contains at least one convergent subsequence.

\textbf{Monotone Convergence Theorem}: Every increasing sequence of real numbers that is bounded above converges. Every decreasing sequence of real numbers that is bounded below converges.

Every real sequence contains either a decreasing subsequence or increasing subsequence or possibly both.

A set is closed if and only if every convergent sequence contained in A has its limit in A.

$\lim\limits_{x \rightarrow x_0} f(x) = y_0$ if and only if for any sequence $\{ x_n\} \in X$ such that $x_n \rightarrow x_0$ and $x_n\neq x_0$, the sequence $\{ f(x_n)\}$ converges to $y_0$.

A function $f$ is continuous at $x_0$ if and only if one of the following equivalent statements is true:
\begin{itemize}
\item $f(x_0)$ is defined and either $x_0$ is an isolated point or $x_0$ is a limit point of X and $\lim\limits_{x \rightarrow x_0} f(x) = f(x_0)$.
\item For any sequence $\{ x_n\}$ s.t. $x_n \rightarrow x_0$, the sequence $\{ f(x_n)\}$ converges to $f(x_0)$.
\end{itemize}

Let $(X,d)$ be a metric space.
\begin{itemize}
\item $\emptyset,X$ are simultaneously open and closed in X;
\item the union of an arbtrary collection of open sets is open;
\item the intersection of a finite collection of closed sets is closed;
\item the union of a finite collection of closed sets is closed;
\item the intersection of an arbitrary collection of closed sets is closed.
\end{itemize}

A function $f$ is continuous iff for any closed set C, the set $f^{-1}(C)$ is closed. A function $f$ is continuous iff for any open set A, the set $f^{-1}(A)$ is open.

\textbf{Supremum Property}: Every nonempty set of real numbers that is bounded above has a supremum, and the supremum is a real number. (Not generally the case for all numbers e.g. sets that would be bounded by irrational numbers in the reals do not have a supremum when they are instead defined in the rationals)

\textbf{Extreme Value Theorem}: Let $f:[a,b]\rightarrow \mathbb{R}$ be continuous. Then $f$ attains its maximum and minimum on $[a,b]$.

\textbf{Intermediate Value Theorem}: Let $f:[a,b]\rightarrow \mathbb{R}$ be a continuous function. Then for any $\gamma \in [f(a),f(b)]$ there exists $c \in [a,b]$ s.t. $f(c) = \gamma$.

Let $f$ be monotonically increasing. Then one-sided limits exist for all x. Moreover, $sup\{f(s)|a<s<x\} = f(x^{-} \leq f(x) \leq f(x^{+}) = inf\{ f(s)|x<s<b\}$.

\textbf{Contraction Mapping Theorem}: Let $(X,d)$ be a nonempty \textbf{complete} metric space and $T:X \rightarrow X$ a contraction with modulus $\beta<1$. Then $T$ has a unique fixed point $x_0$. Additionally, $\forall x_0 \in X$ the sequence $\{ x_n\}$, where $x_n = T^n(x_0) = T(T(\dots T(x_0)\dots ))$ converges to $x_0$.

Continuous dependence of the fixed point on parameters: Let $(X,d)$ and $(\Omega,\rho)$ be two metric spaces and $T:X \times \Omega \rightarrow X.$ For each $\omega \in \Omega$, let $T_{\omega}: X \rightarrow X$ be defined by $T_{\omega}(x) = T(x,\omega)$. Suppose $(X,d)$ is complete, $T$ is continuous in $\omega$, and $\exists \beta < 1$ such that $T_{\omega}$ is a contraction of modulus $\beta$ for all $\omega \in \Omega.$ Then the fixed point function $x^* : \Omega \rightarrow X$ defined by $x^*(\omega) = T_{\omega}(x^*(\omega))$ is continuous.

\textbf{Blackwell's Sufficient Conditions}: Let $B(X)$ be the set of all bounded functions from $X \rightarrow \mathbb{R}$ with metric $d_{\infty}(f,g) = sup_{x\in X} |f(x) - g(x)|$. Let $T:B(X) \rightarrow B(X)$ satisfy:
\begin{itemize}
\item (monotonicity) if $f(x) \leq g(x)$ $\forall x \in X$, then $(T(f))(x) \leq (T(g))(x)$ $\forall x \in X$.
\item (discounting) $\exists \beta \in (0,1)$ s.t. for every $a \geq 0$ and $x \in X$, $(T(f+a))(x) \leq (T(f))(x) + \beta a$.
\end{itemize}
Then $T$ is a contraction with modulus $\beta$.

Any closed subset of a compact space is compact. If $A$ is a compact subset of a metric space, then $A$ is closed and bounded.

\textbf{Heine-Borel Theorem}: If $A \subset \mathbb{R}^m$, then $A$ is compact if and only if $A$ is closed and bounded.

Let $(X,d)$ and $(Y,\rho)$ be metric spaces. If $f: X \rightarrow Y$ is continuous and $C$ is a compact set in $(X,d)$ then $f(C)$ is compact in $(Y,\rho).$

\textbf{Extreme Value Theorem}: If $C$ is a compact set in a metric space $(X,d)$ and $f:C \rightarrow \mathbb{R}$ is continuous, then $f$ is bounded on $C$ and attains its maximum and minimum.

Let $(X,d)$ and $(Y,\rho)$ be metric spaces, $C \subset X$ compact, $f:C \rightarrow Y$ continuous. Then $f$ is uniformly continuous on $C$.

\subsection{Space definitions}

A \textbf{metric} on a set $X$ is a function $d: X $ x $ X \rightarrow \mathbb{R}^+$ s.t. $\forall x,y,z \in X,$
\begin{itemize}
\item $d(x,y) \geq 0$, with equality $\iff x = y;$
\item $d(x,y) = d(y,x)$;
\item $d(x,z) \leq d(x,y) + d(y,z).$
\end{itemize}
A \textbf{metric space} is a pair $(X,d),$ where $X$ is a set and $d$ is a metric on $X$. Examples include Euclidean space.

In a metric space, $(X,d),$ an \textbf{open ball} is $B_\epsilon(x) = \{ y \in X| d(x,y) < \epsilon\}$ and a \textbf{closed ball} is $B_\epsilon[x] = \{ y \in X| d(x,y) \leq \epsilon\}$.

A \textbf{sequence} in a set $X$ is a function $s: \mathbb{N} \rightarrow X,$ denoted $\{ s_n\}$, where $s_n = s(n)$. A sequence $x_n$ in a metric space $(X,d)$ \textbf{converges} to $x \in X$ if $\forall \epsilon >0, \exists N(\epsilon)>0$ s.t. $\forall n>N(\epsilon)$ $d(x_n,x) < \epsilon.$ We write $x_n \rightarrow x$ or $\lim\limits_{n \rightarrow \infty} x_n = x.$

A subset $s \subset X$ in a metric space $(X,d)$ is \textbf{bounded} if $\exists x \in X, \beta \in \mathbb{R} s.t. \forall s \in S, d(x,s) < \beta.$ Every convergent sequence in a metric space is bounded.

Let $(X,d)$ be a metric space. A set $A\subset X$ is \textbf{open} if $\forall x \in A \exists \epsilon > 0 s.t. B_\epsilon(x) \subset A.$ A set $C \subset X$ is \textbf{closed} if its complement is open. This depends on the metric space. $[0,1]$ is not open in $(\mathbb{R},d_E)$ but is open in $([0,1],d_E)$.

Let $(X,d)$, and $A \in X$. $x \in X$ is a \textbf{limit point} of A if $\forall \epsilon > 0,$ $(B_{\epsilon}(x) \setminus \{ x\})\cap A \neq \emptyset.$

Let $(X,d),(Y,\rho)$ be two metric spaces, $A \subset X, f:A \rightarrow Y, x_0 =$limit point of $A$. $f$ has a limit $y_0$ as $x$ approaches $x_0$ if $\forall \epsilon>0 \exists \delta > 0 s.t.$ if $x \in A$ and $0<d(x,x_0) < \delta,$ then $\rho(f(x),y_0) < \epsilon.$

A function is \textbf{continuous} at $x_0$ if $\forall \epsilon>0 \exists \delta>0$ s.t. if $d(x,x_0) < \delta$, then $\rho(f(x),f(x^0)) < \epsilon$. ($\delta$ can vary for different $x^0$ and $\epsilon$)

A function $f$ is continuous if it is continuous at every point of its domain.

A function is uniformly continuous if $\forall \epsilon > 0 \exists \delta>0$ s.t. $if d(x,x_0)<\delta,$ then $\rho(f(x),f(x_0)) < \epsilon$. Note: delta depends only on epsilon!

Let $(X,d),(Y,\rho)$ be two metric spaces, $f:X \rightarrow Y, E\subset X.$ Then f is \textbf{Lipschitz} on $E$ if $\exists K>0$ s.t. $\rho(f(x),f(y)) \leq K d(x,y) \forall x,y \in E.$ $f$ is \textbf{locally Lipschitz} on $E$ if $\forall x \in E \exists \epsilon > 0$ s.t. f is Lipscitz on $B_{\epsilon}(x) \cap E.$

Suppose $X$ is bounded above. The supremum of $X$, $sup X$, is the smallest upper bound for $X$, i.e. $supX$ satisfies
\begin{itemize}
\item $supX \geq x$ $\forall x \in X$;
\item $\forall y<supX \exists x \in X$ s.t. $x>y$.
\end{itemize}
 And infimum is similarly defined.

A sequence is \textbf{Cauchy} if $\forall \epsilon > 0 \exists N > 0$ s.t. if $m,n>N$ then $d(x_n,x_m)<\epsilon.$ Every convergent sequence in a metric space is Cauchy.

A metric space is \textbf{complete} if every Cauchy sequence contained in $X$ converges to some point in $X$. Euclidean space is complete. If $(X,d)$ is a complete metric space and $Y \subset X,$ then $(Y,d)$ is complete if and only if $Y$ is closed.

A function from a metric space to itself is called an \textbf{operator}. An operator is a contraction of modulus $\beta$ if $\beta < 1$ and $d(T(x),T(y)) \leq \beta d(x,y)$ $\forall x,y \in X.$ Every contraction is uniformly continuous.

A collection of sets $U = \{u_{\lambda}|\lambda \in \Lambda \}$ in a metric space is an open cover of the set $A$ if $U_{\lambda}$ is open for all $\lambda \in \Lambda$ and $A \subset \bigcup_{\lambda \in \Lambda} U_{\lambda}.$

A set $A$ in a metric space is \textbf{compact} if every open cover of A contains a finite subcover of $A$. To prove something is compact, find a way to reduce any general open cover into a finite subcover. To prove something is not compact, find an infinite subcover which cannot be reduced in this way.

\section{Linear Algebra}

A \textbf{vector space} $V$ is a collection of vectors, which may be added together and multiplied by scalars, satisfying the following conditions:
\begin{itemize}
\item $\forall x,y,z \in V, (x+y)+z = x+(y+z)$;
\item $\forall x,y \in V, x+y = y+x$;
\item $\exists \vec{0} \in V$ s.t. $\forall x \in V, x + \vec{0} = \vec{0}+x = x;$
\item $\forall x \in V \exists (-x) \in V$ s.t. $x + (-x) = \vec{0};$
\item $\forall \alpha \in \mathbb{R}, x,y \in V, \alpha (x + y) = \alpha x + \alpha y;$
\item $\forall \alpha, \beta \in \mathbb{R}, x \in V, (\alpha + \beta)x = \alpha x + \beta x;$
\item $\forall \alpha, \beta \in \mathbb{R}, x \in V, (\alpha \beta)x = \alpha  (\beta x);$
\item $\forall x \in V, 1 x = x.$
\end{itemize}

A set spans $V$ if $V = \text{span}W$ where a span is a linear combination of a set of vectors.

A set is linearly independent if $\nexists x_1, \dots, x_n \in X, a_1,\dots, a_n \in \mathbb{R}$ s.t. $\sum a_i  \neq 0$ and $\sum a_i x_i = \vec{0}.$

Alternatively, if $a_i x_i = \vec{0}$ implies $a_1 = \dots =a_n  =0 $ then $X$ is linearly independent. A \textbf{basis} is of $V$ is a set of linearly independent vectors in $V$ that spans $V$.

Let $B$ be a basis for $V$ and enumerate elements of $B$ by a set $\Lambda$ so that $B = \{ v_{\lambda} | \lambda \in \Lambda\}.$ Then every vector $x \in V$ has a unique representation as a linear combination of elements of $B$ with finitely many nonzero coefficients.

Every vector space has a basis. Any two bases of a vector space have the same cardinality.

If $V$ is a vector space and $W \subset V$ is linearly independent, then there exists a linearly independent set B such that $W \subset B \subset \text{span}B = V$

Let $V$ be a vector space. The dimension of $V$ is the cardinality of any basis of $V$. dim$R^n = n.$

Suppose dim$V = n \in \mathbb{N}. If W \subset V and |W|>n$, where $|W|$ denotes the cardinality of $W$, then $W$ is linearly dependent.

Suppose dim$V = n$ and $W \subset V, |W| = n$. Then, if W is linearly independent, then span$W = V$, so $W$ is a basis of $V$; if span$W = V$, then $W$ is linearly independent, so $W$ is a basis of $V$.

Let $X$ and $Y$ be two vector spaces. We say that $T:X \rightarrow Y$ is a linear transformation if for all $x_1,x_2 \in X, a_1,a_2 \in \mathbb{R}$, $T(a_1x_1 + a_2x_2) = a_1T(x_1) + a_2T(x_2)$.

If $L(X,Y)$ is the set of all linear transformations from $X$ to $Y$, then $L(X,Y)$ is a vector space.

If $R:X\rightarrow Y, S:Y \rightarrow Z$ then $R \circ S:X \rightarrow Y$ is a linear transformation.

Let $T \in L(X,Y).$ Im$T: \{T(x)| x\\in X \}$ and ker$T:=\{ x \in X | T(x) = \vec{0}\}$, and the rank of T is rank$T := $ dim$($Im$(T)$.

If $T \in L(X,Y)$, then Im$T$ and ker$T$ are vector subspaces of $Y$ and $X$, respectively.

Let X be a finite-dimensional vector space and $T \in L(X,Y).$ \\ Then dim$X = $ dim$($ker$T) + $rank$T = $dim$($ker$T) + $dim$($Im$T).$

$T \in L(X,Y)$ is invertible if there exists a function $S:Y \rightarrow X$ such that $S(T(x)) = x \forall x \in X, T(S(y)) = y \forall y \in Y.$ The transformation $S$ is called the inverse of $T$ and is denoted $T^{-1}$.

$T$ is invertible means:
\begin{itemize}
\item $T$ is one-to-one: $\forall x_1 \neq x_2, T(x_1) \neq T(x_2)$.
\item $T$ is onto: $\forall y \in Y \exists x \in X$ s.t. $T(x) = u$.
\end{itemize}

If $T \in L(X,Y)$ is invertible, then $T^{-1} \in L(Y,X)$

$T \in L(X,Y)$ is one-to-one if and only if ker$T = \vec{0}$.

Two vector spaces $X$ and $Y$ are isomorphic if there exists an invertible linear function from $X$ to $Y$. A function with these properties is called an isomorphism.

Let $X$ and $Y$ be two vector spaces, and let $V$ be a basis for $X$. Then a linear transformation $T: X\rightarrow Y$ is completely defined by its value on $V$:
\begin{itemize}
\item Given any set $\{ y_{\lambda} | \lambda \in \Lambda\} \subset Y$, $\exists T \in L(X,Y)$ s.t. $T(v_{\lambda}) = y_{\lambda}$ for all $\lambda \in \Lambda$.
\item If $S,T \in L(X,Y)$ and $S(v_{\lambda}) = T(v_{\lambda})$ for all $\lambda \in \Lambda$, then $S = T$.
\end{itemize}

Two vectors spaces $X$ and $Y$ are isomorphic iff dim$X = $ dim$Y$. If dim$X = n$, then $X$ is isomorphic to $\mathbb{R}^n$.

Let $V = \{v_1, \dots , v_n \} \in X$ is a basis of $X$. Then $\forall x \in X$ has a unique representation $x = \sum_{i=1}^{n} a_i v_i.$ Then $crd_V(x) = \colvec{3}{a_1}{\dots}{a_n} \in \mathbb{R}^n $ is an isomorphism from $X$ to $\mathbb{R}^n$.

Let $V,W$ be bases of $X,Y$. Then $\forall y \in Y$ has a unique representation $y = \sum_{i=1}^{m}a_iw_i$, e.g. $T(v_1) = \sum_{i=1}^{m} a_{i1}w_i, \dots, T(v_n) = \sum_{i=1}^{m} a_{in}w_i$. Then, \\ 
mtx$_{W,V}(T) =
\begin{pmatrix}
a_{1,1} & a_{1,2} & \dots & a_{1,n} \\
\dots &\dots &\dots &\dots \\
a_{m,1} & a_{m,2} & \dots & a_{m,n}
\end{pmatrix}$ is an isomorphism from $L(X,Y)$ to $M_{m\times n}$.

Example: Let $X = Y = \mathbb{R}^2, V = \{ \colvec{2}{1}{0} \colvec{2}{0}{1}\}, W = \{ \colvec{2}{1}{1} \colvec{2}{-1}{1}\}, T(x) = x$.\\
 Then, mtx$_{W,V} =  
\begin{pmatrix}
0.5 & 0.5 \\
-0.5 & 0.5
\end{pmatrix} = (W^{-1})T(V)$.

mtx$_{W,V} (T) $ mtx$_{V,U}(S) = $ mtx $_{W,U}(T\circ S)$

dim$X = n, T \in L(X,X)$. mtx$_V (T) = $ mtx$_{V,V}(T)$.

mtx$_{V}(T) = P^{-1} $mtx$_{W} (T) P$, where $P = $ mtx$_{W,V}(id)$.

$A,B \in M_{n \times n}$ are similar if $A = P^{-1}BP$ for some invertible matrix $P$.

If dim$X = n$, then
\begin{itemize}
\item if $T \in L(X,X)$, then any two matrix representations of $T$ are similar.
\item two similar matrices represent the same linear transformation $T$, relative to suitable bases.
\end{itemize}
\end{document}
