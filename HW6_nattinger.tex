% !TEX TS-program = pdflatex
% !TEX encoding = UTF-8 Unicode

% This is a simple template for a LaTeX document using the "article" class.
% See "book", "report", "letter" for other types of document.

\documentclass[11pt]{article} % use larger type; default would be 10pt

\usepackage[utf8]{inputenc} % set input encoding (not needed with XeLaTeX)

%%% PAGE DIMENSIONS
\usepackage{geometry} % to change the page dimensions
\geometry{a4paper} % or letterpaper (US) or a5paper or....

\usepackage{graphicx} % support the \includegraphics command and options

\usepackage{amssymb}
\usepackage{amsmath}
%%% PACKAGES
\usepackage{booktabs} % for much better looking tables
\usepackage{array} % for better arrays (eg matrices) in maths
\usepackage{paralist} % very flexible & customisable lists (eg. enumerate/itemize, etc.)
\usepackage{verbatim} % adds environment for commenting out blocks of text & for better verbatim
\usepackage{subfig} % make it possible to include more than one captioned figure/table in a single float
% These packages are all incorporated in the memoir class to one degree or another...

%%% HEADERS & FOOTERS
\usepackage{fancyhdr} % This should be set AFTER setting up the page geometry
\pagestyle{fancy} % options: empty , plain , fancy
\renewcommand{\headrulewidth}{0pt} % customise the layout...
\lhead{}\chead{}\rhead{}
\lfoot{}\cfoot{\thepage}\rfoot{}

%%% SECTION TITLE APPEARANCE
\usepackage{sectsty}
\allsectionsfont{\sffamily\mdseries\upshape} % (See the fntguide.pdf for font help)
% (This matches ConTeXt defaults)

%%% ToC (table of contents) APPEARANCE
\usepackage[nottoc,notlof,notlot]{tocbibind} % Put the bibliography in the ToC
\usepackage[titles,subfigure]{tocloft} % Alter the style of the Table of Contents
\renewcommand{\cftsecfont}{\rmfamily\mdseries\upshape}
\renewcommand{\cftsecpagefont}{\rmfamily\mdseries\upshape} % No bold!

\usepackage{amsmath}
\DeclareMathOperator*{\argmax}{arg\,max}
\DeclareMathOperator*{\argmin}{arg\,min}

\newcount\colveccount
\newcommand*\colvec[1]{
        \global\colveccount#1
        \begin{pmatrix}
        \colvecnext
}
\def\colvecnext#1{
        #1
        \global\advance\colveccount-1
        \ifnum\colveccount>0
                \\
                \expandafter\colvecnext
        \else
                \end{pmatrix}
        \fi
}

\newcommand{\norm}[1]{\left\lVert#1\right\rVert}

\title{HW6}
\author{Michael B. Nattinger\footnote{I worked on this assignment with my study group: Alex von Hafften, Andrew Smith, and Ryan Mather. I have also discussed problem(s) with Emily Case, Sarah Bass, and Danny Edgel.}}

\begin{document}
\maketitle

\section{Question 1}
Bob will travel along the road for some distance $x$, and then turn off the road and travel in the exact direction of "Happy Cow". Bob is minimizing his walking time to reach this point: $min_{x\in [0,12]} x/5 + f(x)/3$ where $f(x)$ is the distance (in miles) through the woods that Bob must travel if Bob chooses to walk $x$ miles on the main road. It can easily be shown via simple geometry that $f(x) = \sqrt{(12-x)^2 + 25}$. Thus, Bob solves the following:
\begin{equation*}
\min_{x\in[0,12]} x/5 + \sqrt{(12-x)^2 + 25}/3.
\end{equation*}
We can take first order conditions of the objective function $g$ with respect to $x$: $\frac{dg}{dx} = 1/5 - \frac{1}{6 \sqrt{(12-x)^2 + 25}}(2(12 - x)) = 0 \Rightarrow 1/5 =  \frac{12 - x}{3 \sqrt{(12-x)^2 + 25}}\Rightarrow (9/25)((12 - x)^2 + 25) = (12 - x)^2 \Rightarrow (9*25)/16 = (12 - x)^2 \Rightarrow (15/4) = (12 - x),-(15/4) = (12 - x)$. If $(12-x)<0$ then $x>12$ so $x\notin [0,12]$, so $(15/4) = (12 - x) \Rightarrow x = 12 - (15/4) \\ \Rightarrow x = (33/4)L.$
\section{Question 2}
% use defn of derivative here. Answer is no.
%Since $f$ is differentiable in $B_{\epsilon}(x_0)$, it is also continuous on that interval. Let $\delta \in (0,\epsilon).$ For $x \in (x_0-\delta,x_0), f^{'}(x) = \lim\limits_{h\rightarrow 0} \frac{f(x+h) - f(x)}{h} <0$ so $\exists h^{*} \in (0,\delta)$ such that $\frac{f(x+h^{*}) - f(x)}{h^{*}}<0$ so $f(x+h^{*}) < f(x)$% show x<x_0.
Assume that $x_0$ is a local maximum of $f$. Then $\exists\delta \in (0,\epsilon]$ such that for any $x\in B_{\delta}(x_0) \setminus \{x_0\}, f(x_0)\geq f(x).$ Then, notice that $x_0-\delta /2 \in  B_{\delta}(x_0).$ Then, by the mean value theorem, $\exists c \in (x_0 - \delta/2,x_0)$ such that $f'(c) = \frac{f(x_0) - f(x_0 - \delta/2)}{\delta/2} >0 $ which is a contradiction, so $x_0$ is not a local maximum of $f$. Now assume that $x_0$ is a local minimum of $f$.  Then $\exists\delta \in (0,\epsilon]$ such that for any $x\in B_{\delta}(x_0) \setminus \{x_0\}, f(x_0)\leq f(x).$ Then, notice that $x_0+\delta /2 \in  B_{\delta}(x_0).$ Then, by the mean value theorem, $\exists c \in (x_0,x_0+\delta/2)$ such that $f'(c) = \frac{f(x_0 + \delta/2) - f(x_0)}{\delta/2} >0 $ which is a contradiction, so $x_0$ is not a local minimum of $f$. 

\section{Question 3}
\begin{align*}
\frac{\partial f}{\partial r} &= \frac{\partial w}{\partial x}\frac{\partial x}{\partial r} + \frac{\partial w}{\partial y}\frac{\partial y}{\partial r} + \frac{\partial w}{\partial z}\frac{\partial z}{\partial r}\\
&= (y^2z)(1) + (2xyz)(2) + (xy^2)(1) \\
&= (2r+4s+t)^2(3r+s+t) + 4(t+2s+t)(2r+3s+t)(3r+s+t) + (t+2s+t)(2r+4s+t)^2, \\
\frac{\partial w}{\partial s} &= \frac{\partial w}{\partial x}\frac{\partial x}{\partial s} + \frac{\partial w}{\partial y}\frac{\partial y}{\partial s} + \frac{\partial w}{\partial z}\frac{\partial z}{\partial s}\\
&= (y^2z)(2) + (2xyz)(3) + (xy^2)(1)\\
&= 2(2r+4s+t)^2(3r+s+t) + 6(t+2s+t)(2r+3s+t)(3r+s+t) + (t+2s+t)(2r+4s+t)^2, \\
\frac{\partial w}{\partial t} &= \frac{\partial w}{\partial x}\frac{\partial x}{\partial t} + \frac{\partial w}{\partial y}\frac{\partial y}{\partial t} + \frac{\partial w}{\partial z}\frac{\partial z}{\partial t}\\
&= (y^2z)(3) + (2xyz)(1) + (xy^2)(1)\\
&= 3(2r+4s+t)^2(3r+s+t) + (t+2s+t)(2r+3s+t)(3r+s+t) + (t+2s+t)(2r+4s+t)^2 .
\end{align*}
\section{Question 4}
Let $f$ be continuously differentiable on $X \subset \mathbb{R}^n$. Then, $f'$ exists and is continuous on $ X.$ Let $x_0 \in X$ and let $B_{\epsilon}(x_0) \subset X$ be a closed epsilon ball around $x_0$. Since $f'$ is continuous, it must be bounded on $B_{\epsilon}(x_0)$. Let $m_1,m_2$ be the upper and lower bounds of $f'$ on $B_{\epsilon}(x_0)$, and let $M = \max\{|m_1|,|m_2| \}$. Then, $|f'(x)|\leq M$ $\forall x \in B_{\epsilon}(x_0).$ Let $x_1,x_2 \in B_{\epsilon}(x_0)$ and assume for the purpose of contradiction that $|f(x_1) - f(x_2)|>M|x_1 - x_2|$. Then, by the mean value theorem $\exists c \in (x_1,x_2)$ such that $f'(c)=\frac{f(x_1) - f(x_2)}{x_2 - x_1},$ and note that this implies $|f'(c)|>M$. $c \in  B_{\epsilon}(x_0)$ so this is a contradiction, and $|f(x_1) - f(x_2)|\leq M|x_1 - x_2|$, so $f$ is locally lipschitz on $X$.
\section{Question 5}
% follow steps from late in notes
% f(0,0) = 0, show det(D_xf(0,0)) \neq 0
% then Dx(a) = -(D_xf(0,0))^{-1}D_af(0,0)
$f(1,1) = 0.$ Det$D_Xf = \text{Det} (5x^4 - 2x +1) \Rightarrow \text{Det}D_Xf(1,1) = 5 - 2+1 \neq 0.$ Then, by the implicit function theorem,
\begin{align*}
\frac{\partial x(y)}{\partial y}|_{(1,1)} &= - \left(\frac{\partial f}{\partial x}|_{(1,1)}\right)^{-1}\left(\frac{\partial f}{\partial y}|_{(1,1)}\right) \\
&= - (5x^4 - 2x + 1 |_{(1,1)})^{-1}(-4y^2 - 2|_{(1,1)})\\
&= - (4)^{-1}(-6) = \frac{3}{2}.
\end{align*}
\section{Question 6}
\begin{align*}
Df(x,y) &= \colvec{2}{8x^3 - y}{2y - x} = \vec{0} \Rightarrow x = 2y, 64y^3 = y \Rightarrow y = 0, y = 1/8, y = -1/8\\
 &\Rightarrow (x,y) = (0,0), (1/4,1/8), (-1/4,-1/8). \\
D^2f(x,y) &= \begin{pmatrix}24x^2 & -1 \\ -1 & 2 \end{pmatrix}. \\
D^2f(0,0) &= \begin{pmatrix}0 & -1 \\ -1 & 2 \end{pmatrix}, D^2f(1/4,1/8) =  \begin{pmatrix}3/2 & -1 \\ -1 & 2 \end{pmatrix} = D^2f(-1/4,-1/8).
\end{align*}
First, we will investigate the point $(0,0)$. Det$(D^2f(0,0) - \lambda I) = 0 \Rightarrow -\lambda(2-\lambda) - 1 = 0 \Rightarrow \lambda^2 -2\lambda -1 = 0 \Rightarrow \lambda =\frac{1}{2} - \frac{\sqrt{5}}{2}, \lambda = \frac{1}{2} + \frac{\sqrt{5}}{2}$. So, one eigenvalue is positive while the other is negative, so $f$ has a saddle point at $(0,0)$.
Next we will investigate the point $(1/4,-1/4).$ Det$(D^2f(1/4,1/8) - \lambda I) = 0 \Rightarrow (3/2-\lambda)(2-\lambda) - 1 = 0 \Rightarrow \lambda^2 - (7/2)\lambda + 2 = 0 \Rightarrow \lambda = \frac{7}{4} + \frac{\sqrt{17}}{4}, \lambda = \frac{7}{4} - \frac{\sqrt{17}}{4}$. Both eigenvalues are positive, so $f$ has a local minimum at $(1/4,1/2)$ and, since Det$(D^2f(1/4,1/8)) = $ Det$(D^2f(-1/4,-1/8))$, $f$ has a local minimum at $(-1/4,-1/2).$
\end{document}
